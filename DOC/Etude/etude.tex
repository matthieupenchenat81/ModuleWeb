\documentclass[12pt,a4paper]{report}
\usepackage[utf8]{inputenc}
\usepackage[francais]{babel}
\usepackage[T1]{fontenc}
\usepackage{amsmath}
\usepackage{amsfonts}
\usepackage{amssymb}
\usepackage{graphicx}
\graphicspath{{pic/}}

\usepackage{authblk}
\author{Quentin Rouand, Jerome Morjon, Matthieu Penchenat, Alexandre Pereira, Sébastien Bouyt}
\affil{Université Toulouse, Jean Jaurès - L3 MIASHS - ProjetWeb}

\begin{document}

\title{Projet Web - AJAX} 
\maketitle

\renewcommand{\contentsname}{Sommaire}
\tableofcontents

\chapter*{Introduction}
\addcontentsline{toc}{chapter}{Introduction}

\chapter{Spécification}

\section{Speech}

Dans le cadre du programme culturel de la mairie de Toulouse, les enfants des écoles maternelles effectuent en 2015 une visite au musée des Augustins. Afin de préparer cette visite, il serait intéressant de proposer des outils numériques visant à introduire de manière ludique et intéractive le musée et ses œuvres auprès des enfants.\\

La Mairie de Toulouse nous a donc demandé de réutiliser des données ouvertes publiés sur son portail dans l'objectif de créer un application web visant un public de 4 à 5 ans accompagné de leur adulte référent. La finalité de l'ensemble de l'application devra être la découverte de la culture par les enfants de maternelle.

Pour ce projet, nous allons devoir réaliser un site responsive qui puisse être adapté sur smartphone, tablette ainsi que sur ordinateur. Ensuite, il faut que ce site soit ergonomique et facile d'utilisation pour un enfant. Il faut qu'il soit intuitif et simple d'utilisation ainsi qu'éducatif et culturel.

- Jeux
- Description adulte référent
- Finalité de l'application


\section{Use Case}

\subsection{Acteurs}
Enfants et adultes référents

\chapter{Questions à poser au client}

\begin{itemize}
\item C'est qui cet adulte référent? 
\item Il choisi des images pour tout le monde?
\item Admin, acteur? 
\item Si c'est côté client ou côté administrateur? 
\item Niveau des enfants 4, 5 ans? (les enfants ne savent pas lire)
\item De quoi disposons comme information?
\end{itemize}

\chapter{Idées}


liste de jeux
jeux 94
jeux du pendu
mélange pour facilité (de jeux)
Mémo et puzzle (compulsory)


%\listoffigures
%\addcontentsline{toc}{chapter}{Table des illustrations}

%\renewcommand{\bibname}{Références}
%\begin{thebibliography}{9}

%\bibitem{lamport94}
 % Arte Reportage (arte futur).
  %Disponible sur http://www.arte.tv/guide/fr/plus7/,
  %2014.

%\end{thebibliography}
%\addcontentsline{toc}{chapter}{Références}

\end{document}
